\documentclass[titlepage,a4paper,12pt]{article}

\usepackage{amsmath,amsfonts,amssymb,amsthm,mathrsfs}
\usepackage[french]{babel}
\usepackage{comment}
\usepackage{tikz}
\usepackage[utf8]{inputenc}
\usepackage[parfill]{parskip}
\newcounter{def}
\newcounter{thm}
\newcounter{prop}
\newcounter{cor}

\newtheorem{interface}[def]{Définition}
\newtheorem{probainv}[thm]{Théorème}
\newtheorem{cdb}[def]{Définition}
\newtheorem{chaine}[prop]{Proposition}
\newtheorem{cut}[def]{Définition}
\newtheorem{bcut}[cor]{Corollaire}
\newtheorem{occ}[def]{Définition}
\newtheorem{bk}[prop]{Proposition}
\newtheorem{decexp}[prop]{Proposition}
\newtheorem{cvg}[thm]{Théorème}
\newtheorem{stc}[prop]{Proposition}
\newtheorem{tension}[prop]{Proposition}


\title{Mémoire M2}
\author{Wei ZHOU}


\begin{document}


\maketitle

\section{Introduction}
Dans ce mémoire, nous allons étudier les interfaces dans le modèle de percolation dans le réseau $\mathbb{Z}^2$ avec une méthode dynamique. Plus généralement, nous considérons un paramètre $p\in [0,1]$ et le réseau $(\mathbb{Z}^d,E^d)$. nous posons $\Omega = \{0,1\}^{E^d}$, $\mathcal{F}\subset \mathcal{P}(\Omega)$ la tribu cylindrique. Nous disons que $\omega = \{\omega(e), e\in E^d\} \in \Omega$ est une configuration pour le réseau $\mathbb{Z}^d$ et que l'arête $e$ est ouverte si $\omega(e) = 1$ et fermée si $\omega(e)= 0$. La probabilité de percolation indépendant de paramètre $p$ est la probabilité produit$P_p = \mathbb{B}_p^{E^d}$, où $\mathbb{B}_p = p\delta_1 +(1-p)\delta_0 $ (la loi de Bernoulli de paramètre $p$).

Chaque configuration correspond à un sous graph de $\mathbb{Z}^d$ si nous considérons uniquement les arêtes ouvertes. Et plus intuitivement, une configuration s'obtient en fermant indépendamment chaque arête du réseau $\mathbb{Z}^d$ avec une probabilité $1-p$. 

Dans notre étude, nous nous concentrons pour instant au cas $d = 2$.

\section{Notations générales}
Dans cette partie, nous définissons d'abord le processus de la percolation dynamique dans le réseau $(\mathbb{Z}^2,E^2)$ puis nous définissons l'interface d'abord à l'aide de la percolation dynamique et d'un couplage. Pour commencer, nous présentons des notations générales pour la suite du mémoire.

Nous notons $\Lambda(l,h)$: le rectangle $[-l,l]\times[-h,h]$;  nous notons aussi les bords $T(l,h)$: les sommets de $\mathbb{Z}^2$ sur le segment $[-l,l]\times(0,h)$ (soit le côté haut de $\Lambda(l,h)$), nous notons aussi $B(l,h)$ les sommets dans $[-l,l]\times(0,-h)$, que nous appelons le coté bas du rectangle et enfin l'ensemble des sommets dans les deux bords verticaux de la boîte que nous notons $V(l,h)$.
\begin{figure}[h]
\center
\begin{tikzpicture}
\draw (0,0) rectangle (8,6);
\node at (4,3) {$\Lambda(l,h)$};
\node[above] at (4,6) {$T(l,h)$};
\node[below] at (4,0) {$B(l,h)$};
\node[left] at (0,3) {$V(l,h)$};
\node[right] at (8,3) {$V(l,h)$};
\end{tikzpicture}
\caption{La boîte $\Lambda(l,h)$};
\end{figure}

Soit $x$ et $y$ deux sommets dans $\mathbb{Z}^2$, nous notons $x\longleftrightarrow y$ s'ils sont reliés par un chemin ouvert, et $x\nleftrightarrow y$ , si ce n'est pas le cas; de plus, soit $B$ un sous ensemble de $E^2$, nous notons $x\overset{B}{\longleftrightarrow} y$ si $x$ et $y$ sont reliés par des arêtes contenant dans $B$;

\paragraph{Graphe dual}Nous rappelons qu'un graphe dual d'une percolation sur le réseau $\mathbb{Z}^2$ est un réseau sur $\mathbb{Z}^2+(\frac{1}{2},\frac{1}{2})$, dont une arête $e^*$ est ouverte ssi l'arête $e$ dans $\mathbb{Z}^2$ qui l'intersecte est ouverte. Désormais, les objets dans le graphe dual seront notés avec une $*$.

\begin{figure}[h]
\center
\begin{tikzpicture}
\draw[gray,dashed] (0.1,0.1) grid (4.9,2.9);
\draw[xshift=0.5cm,yshift=0.5cm] (0,0) grid (4,2);
\draw (2.7,1.7) node {O};
\draw[red,thick] (1.5,0.5) -- (1.5,1.5);
\draw[red,thick,dashed] (1,1)--(2,1);
\end{tikzpicture}
\caption{un réseau $\mathbb{Z}^2$ et son dual (pointillée)}
\end{figure}

\paragraph{Ensemble séparant}Nous introduisons la notion d'un ensemble séparant, soit $A,B$ deux sous ensemble de $\mathbb{Z}^2$, nous disons qu'un ensemble d'arête $S\subset E^2$ sépare $A$ et $B$ si pour tout sommet $x\in A$ et tout sommet $y\in B$, $x\overset{E^2\setminus S}{\nleftrightarrow} y$. $S$ est appelé un ensemble séparant pour $A$ et $B$.

Nous disons que $S$ est un ensemble séparant minimal de $A,B$ si $S$ sépare $A,B$ et pour tout arête $e$ de $S$, il existe $x\in A$, $y\in B$ tel que $x\overset{E^2\setminus (S\setminus \{e\})}{\longleftrightarrow} y$.
\section{La percolation dynamique}
Nous définissons d'abord le processus la percolation dynamique $$X: \mathbb{R}^+ \rightarrow \{0,1\}^{E^2}$$ dans un réseau $(\mathbb{Z}^2,E^2)$. 
Soit $p\in [0,1]$, et soit $e\in E^2$, notons $X_t$ la configurations de $E^2$ à l'instant $t$ et $X_t(e)$ la configuration d'une arête $e$ à l'instant $t$. Nous écrivons que $X_t(e)= 1$ si $e$ est ouverte à l'instant $t$ et $X_t(e)=0$ sinon. Nous associons à $e$ indépendamment des autres arête un processus markovien de sauts à deux états $\{0,1\}$ qui représente le changement de configuration de l'arête. Le taux de saut dépend de la configuration de $e$:
$$\lambda(X_t,e) = \left\lbrace \begin{array}{lc}
1-p & X_t(e) = 1 \\
p & X_t(e) = 0
\end{array}
\right..
$$ 
Nous avons une définition alternative suivante: nous associons à l'arête $e$ un processus de Poisson de paramètre 1 $N_t(e)$. Nous notons $T_1(e),T_2(e),\dots$ les instants de sauts de $N_t(e)$, à chaque $T_i(e)$, nous retirons la configuration de $e$ selon une loi de Bernoulli de paramètre $p$. Nous voyons bien qu'avec cette définition, la probabilité $P_p$ est l'unique probabilité stationnaire du processus.

Nous pouvons aussi définir $(\tau_i)_{i\geqslant 0}$ la suite des instants de changement de configuration par récurrence dans une boîte finie $\Lambda(l,h)$: soit $$\left.
\begin{array}{l}
\tau_0 = 0 \\
\forall i\in \mathbb{N},\tau_{i+1} = \inf \{ t>\tau_i, \exists e_{i+1} \in \Lambda(l,h), X_t(e) \neq X_{\tau_i}(e)\}
\end{array}\right.,$$
et nous appelons la suite $(e_{i+1})_{i\in\mathbb{N}}$ la suite d'arête modifiée.
\section{L'interface}
Nous définissons l'interface $\mathcal{I}$ à l'aide de la percolation dynamique. Nous définissons d'abord un couplage suivant:

Soit $X_t$ le processus de percolation dynamique de paramètre $p$ dans $\Lambda_{l,h}$ qui satisfait $T\overset{X_0}{\nleftrightarrow}B$ et $(\tau_i)_{i\geqslant 0}$ sa suite des instants de changement de configuration, soit $Y_t$ un processus à valeurs dans $\{0,1\}^E$. Nous déterminons $Y_t$ par récurrence sur $i$: $Y_0=X_0$; soit $i\geqslant 0$, pour tout $s\in [\tau_i, \tau_{i+1}[$, $Y_s = Y_{\tau_i}$, soit $\bar{Y}_i$ la configuration où nous remplaçons $Y_{\tau_i}(e_{i+1})$ par $Y_{\tau_{i+1}}(e_{i+1})$ dans $Y_{\tau_i}$, nous déterminons $(X_{\tau_{i+1}},Y_{\tau_{i+1}})$ ainsi:
$$(X_{\tau_{i+1}}(e),Y_{\tau_{i+1}}(e))=\left\lbrace \begin{array}{cc}
(X_{\tau_{i+1}}(e),X_{\tau_{i+1}}(e)) & e \neq e_{i+1}\\
(X_{\tau_{i+1}}(e_{i+1}),X_{\tau_{i+1}}(e)_{i+1}) & e = e_{i+1}, T \overset{\bar{Y}_i}{\nleftrightarrow} B \\
(1,0) & e = e_{i+1},T\overset{\bar{Y}_i}{\leftrightarrow} B 
\end{array}\right..$$

Nous remarquons d'abord qu'une arête $e$ devient différente dans les deux configurations à l'instant $\tau_i$ si et seulement si $e$ devient ouverte à $\tau_i$ mais cette ouverture induit une connexion entre $T$ et $B$ dans la configuration $\bar{Y}_{i-1}$; dans ce cas, l'arête $e$ reste fermée dans la configuration $Y_t$. 

La chaîne de Markov $(X_{\tau_i})_{i\geqslant 0}$ est irréductible et à l'espace d'états fini donc elle admet une probabilité invariante qui est la probabilité de la percolation Bernoulli de paramètre $p$. La chaîne $(Y_{\tau_i})_{i\geqslant 0}$ est aussi irréductible car toute configuration de $(Y_{\tau_i})_{i\geqslant 0}$ est reliée à la configuration où toutes les arêtes sont fermées. En effet, pour toute arête $e$ dans $\Lambda(l,h)$, et à toute instant de saut $(T_i(e))_{i\geqslant 0}$, on peut choisir de fermer $e$ avec probabilité $1-p$ qui est strictement positive. Le théorème suivant décrit la probabilité invariante de $Y$.

\begin{probainv}
Soit $Z_n$ une chaîne de Markov à valeurs dans un ensemble $E$ fini, irréductible de probabilité invariante $\pi$, soit $A\subset E$, nous définissons $Z_n^A$ par sa probabilité de transition:
$$p^A(x,y)=\left\lbrace \begin{array}{cc}
p(x,y) & y\in A, x\neq y \\
0 & y\notin A \\
1-\sum_{y\neq x}p^A(x,y) & x = y\\
\end{array}
\right.
$$
alors $Z^A$ admet une probabilité invariante qui est $\frac{\mathbf{1}_A(\centerdot)\pi(\centerdot)}{\pi(A)}$ soit la probabilité invariante de la chaîne conditionnée à rester dans $A$.
\end{probainv}

Nous appliquons le théorème à $(Y_{\tau_i})_{i\geqslant 0}$ avec $A$ l'ensemble des configuration qui satisfont $T\nleftrightarrow B$. Nous obtenons la probabilité invariante qui est la loi de la percolation Bernoulli de paramètre $p$ conditionnée par l'événement $T\nleftrightarrow B$.

\begin{interface}
Soit $(X_t,Y_t)_{t\geqslant 0}$ un couplage défini précédemment, nous appelons une interface dans $\Lambda({l,h})$ notée $\mathcal{I}_t({l,h})$ l'ensemble aléatoire des arêtes qui sont ouvertes dans $X_t$ et fermé dans $Y_t$: $$ \mathcal{I}_t({l,h}) = \big\{ e\in E: X_t(e) = 1, Y_t(e) = 0 \big\}.
$$
\end{interface}
\section{Condition aux bords}
Nous allons étudier l'influence de $l$ sur la loi de l'interface dans la boîte $\Lambda(l,h)$. Nous introduisons la notion de conditions aux bords. Pour instant, nous traitons uniquement le cas où seulement $l$ varie. Nous fixons alors un entier $h$ et nous introduisons l'ensemble $\{top,bot,null\}$. Nous notons $\partial^{in}\Lambda(l,h)$ le bord intérieur de $\Lambda(l,h)$, soit
$$ \partial^{in}\Lambda(l,h) = \{x\in \Lambda(l,h), \exists y \notin \Lambda(l,h), \langle x,y\rangle\in E^2\}.
$$
Une condition aux bord $\Pi_{l,h}$ est une application de $\partial^{in}$ à $\{top,bot,null\}$ qui se définit en deux étapes.
Nous posons d'abord la condition aux bords sur les bords horizontaux $T(l,h)$ et $B(l,h)$:
$$ \forall t\geqslant 0, \forall x\in T\cup B, \Pi_t^{l,h}(x) = \left\lbrace \begin{array}{ll}
top & x\in T\\
bot & x\in B
\end{array}
\right.
$$ et nous appelons une condition aux bords nulle si
$$\forall y\in V(l,h)\setminus \{(\pm l,\pm h))\}, \Pi(y) = null.
$$

Soit $t\geqslant 0$, pour un sommet $x$ sur le bord vertical $V(l,h)$, nous déterminons $\Pi_{l,h}(x)$ à l'aide une boîte plus grande de condition aux bords nulle. Plus précisément, une condition aux bords $\Pi_{l,h}$ est une application tel qu'il existe $m>l$ un entier naturel et une boîte $\Lambda(m,h)$ de condition aux bords nulle qui contient la boîte $\Lambda(l,h)$ de configuration $X$ qui vérifie:
$$\forall x \in V(l,h),\Pi_{l,h}(x)=\left\lbrace \begin{array}{cl}
top & \Leftrightarrow x\overset{X}{\underset{\Lambda(m,h) \setminus \Lambda(l,h)}{\longleftrightarrow}} T\\
bot & \Leftrightarrow x\overset{X}{\underset{\Lambda(m,h) \setminus \Lambda(l,h)}{\longleftrightarrow}} B\\
null & \Leftrightarrow x\overset{X}{\underset{\Lambda(m,h) \setminus \Lambda(l,h)}{\nleftrightarrow}} T\cup B 
\end{array} \right.,
$$
la notation $x\overset{\Lambda(m,h) \setminus \Lambda(l,h)}{\longleftrightarrow} T(resp. B)$ signifie que le sommet $x$ est relié à $T(resp. B)$ uniquement par un chemin ouvert dont les arrêtes sont dans l'ensemble $\big(\Lambda(m,h) \setminus \Lambda(l,h)\big) \cup \partial^{in} \Lambda(l,h)$. 

\begin{figure}[h]
\center
\begin{tikzpicture}
\draw (-6,-2) rectangle (6,2);
\filldraw[fill=gray] (-4,-2) rectangle (4,2);
\draw[rounded corners] (-4,0.5) .. controls (-4.5,0) and (-4.3,0.5) .. (-5.5,1) 
			   .. controls (-5.7,1.2) .. (-5,2);
\draw (-3.7,0.5) node {$top$};
\draw (0,2.3) node {$T(l,h)$};
\draw (0,-2.3) node {$B(l,h)$};
\draw (0,0) node {$\Lambda(l,h)$};
\draw (-5,0) node {$\Lambda(m,h)\setminus \Lambda(l,h)$};
\draw[rounded corners] (-4,-0.5) .. controls (-4.5,0) and (-4.3,-1.5) ..(-5,-2);
\draw (-3.7,-0.5) node {$bot$};
\draw[rounded corners] (4.2,2) .. controls (3.5,1) and (3,.5) ..(4,0)
			  .. controls (4.5,-0.3) and (3.1,-1) .. (5,-2);
\draw (3.7,-1.2) node {$bot$};
\draw (3.6,-0.5) node {$null$};
\draw (4.4,0.2) node {$null$};
\draw (4.3,1.6) node {$top$};
\end{tikzpicture}
\caption{Exemples d'une condition aux bords}
\end{figure}

Nous pouvons aussi déterminer la condition aux bords d'une boîte $\Lambda(l,h)$ induite par une boite $\Lambda(m,h)$ de configuration $X$ avec une condition aux bords $\Pi_{m,h}$ non nulle. Soit $x$ un sommet dans $V(l,h)$, 
$$\Pi_{l,h}(x)=\left\lbrace \begin{array}{cl}
top & \exists y \in \partial^{in}\Lambda(m,h), \Pi_{m,h}(y) = top, x\overset{X}{\underset{\Lambda(m,h) \setminus \Lambda(l,h)}{\longleftrightarrow}} y\\
bot & \exists y \in \partial^{in}\Lambda(m,h), \Pi_{m,h}(y) = bot, x\overset{X}{\underset{\Lambda(m,h) \setminus \Lambda(l,h)}{\longleftrightarrow}} y\\
null & \forall y \in \partial^{in}\Lambda(m,h), \Pi_{m,h}(y) \neq null, x\overset{X}{\underset{\Lambda(m,h) \setminus \Lambda(l,h)}{\nleftrightarrow}} y
\end{array} \right..
$$

Nous généralisons désormais l'événement $T\leftrightarrow B$ avec la condition aux bords. Nous disons $T\leftrightarrow B$ s'il existe une connexion entre un sommet $x$ tel que $\Pi_{l,h}(x)= top$ (resp. $bot$) et $B(l,h)$ (resp. $T(l,h)$), de plus nous disons $T\leftrightarrow B$ s'il existe une connexion entre deux sommets $x,y$ tels que $\Pi_{l,h}(x) = top, \Pi_{l,h}(y) = bot$.
\section{Quelques résultats géométriques}
Nous nous intéressons à la probabilité qu'une arrête fixée soit dans l'interface.Nous considérons désormais le régime surcritique, qui correspond au cas $p>\frac{1}{2}$ en dimension 2. Soit $e_0$ l'arête qui joint $(0,0)$ et $(1,1)$, notre objectif est de montrer que à $t$ fixé la probabilité $\mathbb{P}(e_0 \in I^l_t)$ converge lorsque $l$ tend vers l'infini.

Intuitivement, nous imaginons que l'influence d'une condition aux bords disparaît quand la taille de la boîte grandit. %Plus précisément, nous considérons un couplage de processus de percolation dynamique conditionné à la condition $T\nleftrightarrow B$ dans deux boîtes de taille $l$ munies de deux conditions aux bords qui diffèrent en un seul sommet $v_0$. La proposition suivante décrit la façon dont cette différence au bord induit une différence à l'arête $e_0$.

Soit $\Lambda(l,h)$ une boite dans $\mathbb{Z}^2,E^2$ munis de deux conditions aux bords $\Pi_{l,h}, \Pi'_{l,h}$ qui diffèrent à un seul sommet $v_0$. Nous considérons $(X_t)_{t\geqslant 0}$ un processus de percolation dynamique dans $\Lambda(l,h)$ et deux couplage $(X,Y)$ et $(X,Y')$ conditionnés par $T\nleftrightarrow B$ avec les deux conditions aux bords.

\begin{chaine}
\label{chaine}
Soit $t>0$ premier instant où $e_0$ diffère dans les deux configurations, alors il existe $N\in \mathbb{N},$ une suite d'arête $\varepsilon_1,\dots,\varepsilon_N$ et des instant $0<t_1< t_2<\dots<t_N=t$ qui vérifient les conditions suivantes:

\begin{itemize}
\item[(i)] à l'instant $t_1$, $\varepsilon_1$ diffère dans les deux configurations. Il existe deux chemins fermés $p_1^*,q_1^*$ disjoints qui relient $\varepsilon_1^*$ et le dual du bord contenant $v_0$;

\item[(ii)] pour tout $i>1$, l'arête $\varepsilon_i$ devient différente à $t_i$ et il existe deux chemins fermés $p_i^*,q_i^*$ disjoints qui relient $\varepsilon_{i-1}^*$ et $\varepsilon_i^*$.

\item[(iii)] pour tout $i\geqslant 1$, à l'instant $t_i$, il existe un ensemble d'arêtes fermées $S_i$ qui sépare $T(l,h)$ et $B(l,h)$ et un chemin fermé $c^*_i$ disjoint de $S_i$ tel que $c_i^* \subset (p_i^*\cup q_i^*)$ et $\vert c_i^*\vert \geqslant \frac{1}{2} \vert p_i^*\cup q_i^* \vert$
\end{itemize}

\end{chaine}
\begin{proof}
Nous notons $\tau_0 = 0$, $\tau_i = \inf \{s > \tau_{i-1}, \exists e \in \Lambda(l,h),Y_{s^-}(e) = Y'_{s^-}(e),Y_s(e) \neq Y'_s(e)\}$. Nous considérons maintenant la première arête $e_1$ qui devient différente dans les deux boites. Par symétrie, nous pouvons supposer que $v_0$ soit dans le bord gauche et que $e_1$ soit fermée dans $\Lambda'_l$.

Cette arête devient différente car il si elle était ouverte dans la configuration $Y'$, il existerait un chemin ouvert entre $T$ et $B$ et pas dans la configuration $Y$. Il existe alors un chemin ouvert entre $e_1$ et le sommet $v_0$ au bord. Sans perte de généralité, nous supposons que $\Pi'_l(v_0)=T$, il existe alors dans $Y_{\tau_1}$ et $Y'_{\tau_1}$ un chemin identique qui relie une extrémité de $e_0$ et $B$. Nous considérons maintenant le cluster ouvert de ce chemin dans la configuration $Y'_{\tau_1}$, il n'est pas connecté à $B$ par la condition $T\nleftrightarrow B$ dans $Y'_{\tau_1}$, de plus, il n'est pas connecté à $T$ car s'il l'était, $T\longleftrightarrow B$ dans la configuration $Y{\tau_1}$ où $e_0$ est ouverte. Il existe donc dans le graphe dual un contour fermé $C^*_1$ qui sépare ce cluster de $T$ et de $B$. L'arête $e^*_1$ sépare donc ce contour en deux parties disjointes (voir Figure \ref{fig:e1}). 
\begin{figure}[h]
\begin{minipage}{0.45\linewidth}
\center
\begin{tikzpicture}
\draw (2.5,-0.5) node {$Y_{\tau_1}$};
\draw (5,0) --(0,0) -- (0,4) -- (5,4);
\node[fill,circle,inner sep = 1pt] at (2,3) (e+) {};
\node[fill,circle,inner sep = 1pt] at (2,2.8) (e-) {};
\draw (e+) -- (e-);
\node[fill,circle,inner sep = 1pt, label= left:$v_0$] at (0,3.2) {};
\draw (2.3,2.9) node{$e_0$};
\draw (0,3.2) .. controls (1,2.7) .. (e+);
\draw (e-) .. controls (4,2) and (3,1.2) .. (4,0);
\draw[dashed] (2,4) .. controls (0.5,3.5) .. (1,2.8);
\draw[dashed, red, rounded corners] (0,2.2) .. controls (1,1.5) and (1.5,2.7).. (2,2.9)
					.. controls (3,4) .. (0,3.5);
\end{tikzpicture}
\end{minipage}
\begin{minipage}{0.45\linewidth}
\center
\begin{tikzpicture}
\draw (2.5,-0.5) node {$Y'_{\tau_1}$};
\draw (5,0) --(0,0) -- (0,4) -- (5,4);
\node[fill,circle,inner sep = 1pt] at (2,3) (e+) {};
\node[fill,circle,inner sep = 1pt] at (2,2.8) (e-) {};
\node[fill,circle,inner sep = 1pt, label= left:$v_0(T)$] at (0,3.2) {};
\draw (2.3,2.9) node{$e_0$};
\draw (0,3.2) .. controls (1,2.7) .. (e+);
\draw (e-) .. controls (4,2) and (3,1.2) .. (4,0);
\draw[dashed] (2,0) .. controls (0.5,1.5) .. (1,2.8);
\draw[dashed, red, rounded corners] (0,2.2) .. controls (1,1.5) and (1.5,2.7).. (2,2.9)
					.. controls (3,4) .. (0,3.5);
\end{tikzpicture}
\end{minipage}
\caption{La configuration dans les deux boites quand $e_1$ devient différente et le contour $C_1$ en rouge}
\label{fig:e1}
\end{figure}

Nous considérons maintenant une arête $e_i$ qui devient différente à cause des autres arêtes qui sont déjà différentes. Par symétrie, nous supposons que $e_i$ est ouverte dans $Y_{\tau_i}$ et fermée dans $Y'_{\tau_i}$ et que dans les deux boîtes et $e_i$ est reliée à $B$.

Nous pouvons choisir une arête $e_j$ avec $j< i$ qui est devenue différente avant $\tau_i$ et qui est restée différente entre $\tau_j$ et $\tau_i$. En plus, il existe un chemin ouvert qui les relie. En effet, nous considérons le cluster ouvert $C_i$ dans $Y_{\tau_i}$ du sommet qui est séparé de $B$ par l'arête $e_i$. Nous considérons $R_i$ l'ensemble suivant:
$$ R_i = \left\lbrace e = \langle x,y\rangle, x\in C_i, y\notin C_i, y\underset{Y'_{\tau_i}}{\leftrightarrow} T\cup \Pi_{l,h}^{-1}(top), Y_{\tau_i}(e)\neq Y'_{\tau_i}(e) \right\rbrace.
$$
Cet ensemble n'est pas vide, car $e_i\overset{Y'_{\tau_i}}{\leftrightarrow} T$, mais pas dans $Y_{\tau_i}$, donc il existe un tel $\langle x,y\rangle$ sur le bord extérieur de $C_i$ qui diffère dans les deux configurations. Nous pouvons ordonner cet ensemble avec l'ordre lexicographique de $\mathbb{Z}^2$ et nous prenons $e_j$ la première arête de $R_i$.

Comme pour la première arête $e_0$, il y a un chemin ouvert qui relie $e_i$ et $e_j$. Sans perte de généralité, nous supposons que $e_j$ est reliée à $T$.
Nous pouvons considérer le cluster ouvert du chemin $c$ entre $e_i,e_j$ dans $Y'_{\tau_i}$ arrêté à $e_j$ et $e_i$. Ce cluster n'est pas connecté à $T$ car $e_i$ est ouverte, il n'est pas connecté à $B$ dans $Y'_{\tau_i}$ car $e_j$ est ouverte. Nous pouvons donc trouver un ensemble $S_i$ d'arêtes fermées qui sépare ce chemin de $T$ dans $Y_{\tau_i}$ et un ensemble $S'_i$ qui le sépare de $B$ dans $Y'_{\tau_i}$. Nous pouvons supposer que $S_i\cap S'_i \neq \emptyset$, sinon, nous avons deux ensembles qui séparent $T$ et $B$ dans $\Lambda(l,h)$ qui est de probabilité inférieure à $e^{-\lambda l}$ et nous avons le résultat voulu. Nous pouvons désormais considérer un ensemble $S*$ formé par des arêtes qui sont soit fermées dans l'une des deux configurations et qui sépare le chemin $c$ de $T$ et de $B$. Quitte à prendre prendre l'ensemble minimal, nous pouvons supposer que $S^*$ est un circuit qui relie $e_i$ et $e_j$. Les deux arêtes $e_i$ et $e_j$ coupent cet ensemble en deux chemins fermés $p_i^*, q_i^*$.
\begin{figure}[h]

\begin{minipage}{0.45\linewidth}
\center
\begin{tikzpicture}
\draw (2.5,-0.5) node {$Y_{\tau_i}$};
\draw (5,0) --(0,0);
\draw (0,4) -- (5,4);
\node[fill,circle,inner sep = 1pt] at (2,3) (ej+) {};
\node[fill,circle,inner sep = 1pt] at (2,2.8) (ej-) {};
\node[fill,circle,inner sep = 1pt] at (3.5,1) (ei+) {};
\node[fill,circle,inner sep = 1pt] at (3.5,0.8) (ei-) {};
\draw (ei+) -- (ei-);
\node at (2,2.9) [right]  (ej) {$e_j$};
\node[right] at (3.5,.9) (ei) {$e_i$};
\draw (1,4) .. controls (1.5,3) .. (ej+);
\draw (ej-) .. controls (3,2) .. (ei+);
\draw (ei-) .. controls (2,0.5) .. (2.5,0);
\draw[dashed] (4,4) .. controls (4,3) .. (3,2);
\draw[dashed, red, rounded corners] (ej) .. controls (1,2.5) and (1.5,0.7).. (ei)
					.. controls (4,1.4) and (3.5,2.5).. (ej);
\end{tikzpicture}
\end{minipage}
\begin{minipage}{0.45\linewidth}
\center
\begin{tikzpicture}
\draw (2.5,-0.5) node {$Y'_{\tau_i}$};
\draw (5,0) --(0,0);
\draw (0,4) -- (5,4);
\node[fill,circle,inner sep = 1pt] at (2,3) (ej+) {};
\node[fill,circle,inner sep = 1pt] at (2,2.8) (ej-) {};
\node[fill,circle,inner sep = 1pt] at (3.5,1) (ei+) {};
\node[fill,circle,inner sep = 1pt] at (3.5,0.8) (ei-) {};
\draw (ej+) -- (ej-);
\node at (2,2.9) [right]  (ej) {$e_j$};
\node[right] at (3.5,.9) (ei) {$e_i$};
\draw (1,4) .. controls (1.5,3) .. (ej+);
\draw (ej-) .. controls (3,2) .. (ei+);
\draw (ei-) .. controls (2,0.5) .. (2.5,0);
\draw[dashed] (1,0) .. controls (2.5,2) .. (3,2);
\draw[dashed, red, rounded corners] (ej) .. controls (1,2.5) and (1.5,0.7).. (ei)
					.. controls (4,1.4) and (3.5,2.5).. (ej);
\end{tikzpicture}
\end{minipage}

\caption{La configuration dans les deux boites quand $e_i$ devient différente avec le contour $C_i$ en rouge}
\label{fig:ei}
\end{figure}

Enfin, pour obtenir la suite $(\varepsilon_i)_{i\geqslant 1}$ dans l'énoncé, nous partons de l'instant $t$ pour trouver l'arête qui a causé la différence à $e_0$, ensuite nous répétons ce procédure à l'instant où cette arête est devenue différente jusqu'à arriver au bord. Comme chaque ouverture ou fermeture d'une arête est donnée par une horloge exponentielle, il y a presque sûrement un nombre fini d'arêtes qui sont devenues différentes avant $t$. nous pouvons donc trouver une suite finie d'arêtes $\varepsilon_0,\varepsilon_1,\dots, \varepsilon_N$ avec $\varepsilon_N = e_0$ et des instants $t_1<\dots< t_N = t$ qui vérifient les deux premiers points de l'énoncé.

Nous considérons le contour fermé formé par $p_i^1$ et $p_i^2$, la condition $T\nleftrightarrow B$ impose qu'il existe un ensemble séparant $K_i$ qui contient $\varepsilon_i$ dans la configuration où elle est fermée. Quitte à ouvrir les arêtes de $K_i$ qui ne sont pas pivotes, nous pouvons supposer $K_i$ minimal. Nous considérons le dual de $K_i$, qui est un chemin fermé simple de gauche à droite. Nous pouvons numéroter les arêtes de $K_i$ de la façon suivante:
\begin{itemize}
\item nous posons $\varepsilon^*_i = k_0$;
\item si $\varepsilon^*_i$ est horizontale, nous posons $k_{-1}$ l'arête qui partage le sommet gauche de $\varepsilon^*_i$ et $k_1$ l'arête qui partage le sommet droite. S'il $\varepsilon_i^*$ est verticale, nous posons $k_{-1}$ l'arête en bas et $k_1$ celle en haut.
\item Nous nous éloignons de $\varepsilon_i$, jusqu'à avoir numéroté toutes les arêtes. Dans la direction négative, nous diminuons l'indice de 1 à chaque pas et nous augmentons de 1 par pas dans la direction positive.
\end{itemize}

Nous considérons maintenant l'arête minimale et l'arête maximale de l'ensemble $K_i\cap (p_i^1\cup p_i^2)$ que nous notons $k_{min}$ et $k_{max}$. Les deux arêtes $k_{min}$ et $k_{max}$ coupe $K_i$ en 3 parties disjointes: $C^g_i$ un chemin entre le bord gauche et $k_{min}$; $C^d_i$ un chemin entre le bord droite et $k_{max}$; l'union d'un chemin entre $k_{min}$ et $k_{max}$ et ces deux arêtes. De plus, les deux sommets qui séparent ces trois parties coupent le contour $p_i^1\cup p_i^2$ en deux chemins disjoints $c_i^1$ et $c_i^2$. Quitte à échanger les numéros, nous supposons que $|c_i^1| \geqslant |c_i^2|$.
\begin{figure}[h]
\center
\begin{tikzpicture}
\draw (0,0) rectangle (8,4);
\node[left] at (3,2) {$k_{min}$};
\node[above] at (5,2) {$k_{max}$};
\node[fill,circle,inner sep = 1pt] at (3,1.9) (xi) {};
\node[fill,circle,inner sep = 1pt] at (3,2.1) {};
\node[fill,circle,inner sep = 1pt] at (5,2.1) (yi) {};
\node[fill,circle,inner sep = 1pt] at (5,1.9) {};
\draw (0,2) .. controls (1,1) .. (xi) --(3,2.1)
		.. controls (3.5,2.5) and (4.5,2.1) .. (yi)
		.. controls (7,1.3) .. (8,3);
\draw[red] (xi) .. controls (3.5,0.5) and (4.2,1.5) .. (5,1.9) -- (yi);
\node at (4,1) {$c_i^*$};
\node at (1.5,1) {$C_i^g$};
\node at (7,2) {$C_i^d$};
\end{tikzpicture}
\caption{construction d'un chemin fermé disjoint d'un cut}
\end{figure}

Enfin nous posons $C^*_i = C^g_i \cup C^d_i \cup c_i^2$ un chemin fermé du bord gauche au bord droite dans le graphe dual et $c_i^* = c_i^1$. nous posons $C_i$ le dual de $C^*_i$ qui est un cut. Or $|c_i^1| \geqslant |c_i^2|$, nous obtenons $|c^*_i| \geqslant \frac{1}{2}|p_i^1\cup p_i^2|$. Par construction, $C_i$ et $c_i^*$ sont disjoints.
\end{proof}

\section{Quelques inégalités préliminaires}
nous allons énoncer quelques inégalités utiles pour montrer la convergence en loi de l'interface quand la taille de la boîte tend vers infini. Nous commençons par une inégalité de type BK adaptée au problème. Nous commençons par généraliser la notion de l'occurrence disjointe.

\begin{occ}
Soit $\Lambda(l,h)$ une boîte de longueur $l$, soit $A$ l'événement $x\longleftrightarrow y$ à l'instant $s$, $B$ l'événement $u\longleftrightarrow v$ à l'instant $t$, soit $K(\omega) = \{e: e=1\}$, nous définissons l'occurrence disjointe de $A$ et $B$ noté par $A\circ B$ par:
\begin{itemize}
\item si $s=t$, $A\circ B = \{\omega: \exists\omega_1\in A,\exists\omega_2\in B, K(\omega_1)\cap K(\omega_2) = \emptyset, K(\omega_1)\cup K(\omega_2)\subset K(\omega) \}$;
\item si $s<t$, $A\circ B = \{\omega: \exists\omega_1\in A,\exists\omega_2\in B, K(\omega_1)\cup K(\omega_2)\subset K(\omega), \forall e\in (\omega_1)\cap K(\omega_2), \exists r\in [s,t], e(r) = 1 \}$.
\end{itemize}
\end{occ}
Nous énonçons l'inégalité de BK pour la percolation dynamique:

\begin{bk}
Soit $\Lambda(l,h)$ une boîte de longueur $l$, $0<s\leqslant t$, soit $A$ l'événement $x\longleftrightarrow y$ à l'instant $s$, $B$ l'événement $u\longleftrightarrow v$ à l'instant $t$, alors $P(A\circ B) \leqslant P(A)P(B)$ quand $p$ est assez proche de 1.
\end{bk}

\begin{proof}
Pour le premier cas, c'est l'inégalité de BK classique. Nous traitons uniquement le deuxième cas où les instants sont différents. Soit $\Gamma = \{f:[s,t]\rightarrow\{0,1\}\text{ càdlàg} \}^{|\Lambda(l,h)|}$. Nous introduisons deux espaces de probabilité identiques $S_1 = (\Gamma_1,\mathcal{F}_1,P_1), S_2 = (\Gamma_2,\mathcal{F}_2,P_2)$, nous définissons $S$ l'espace produit de $S_1,S_2$. Nous écrivons $x\times y$ un point de $\Gamma_1 \times \Gamma_2$. Nous notons $A' = A\times \Gamma_2$, $B_k' = \{x\times y :(y_1,\dots,y_k,x_{k+1},\dots,x_{|\Lambda(l,h)|}) \in B\}$. Nous notons $P_{12}$ la probabilité produit de $P_1,P_2$. Nous avons bien que $P(A\circ B) = P_{12}(A'\circ B'_0)$ et $P_{12}(A'\circ B'_{|\Lambda(l,h)|}) = P(A)P(B)$. Nous montrons maintenant: $$\forall k>0, P_{12}(A'\circ B'_{k-1}) \leqslant P_{12}(A'\circ B'_k)$$. 

Nous considérons $x\times y \in A'\circ B'_{k-1}$, donc $x\in A$ et $(y_1,\dots,y_{k-1},x_k,\dots,x_{|\Lambda(l,h)|})\in B$. 

Nous considérons d'abord le cas où $e_k$ n'est pas pivot pour $B$. Nous vérifions facilement $x\times y \in A'\circ B'_k$.

Nous considérons maintenant l'arête $e_k$ est pivot pour $A$ et $B$. Nous devons avoir $x_k(s) = 1, x_k(t) =1, \exists r\in [s,t], x_k(r) = 0$ pour $A\circ B_{k-1}$. Pour $A\circ B_k$, il faut $x_k(s)=1$ et $y_k(t)=1$. Nous montrons que $$P\big(x_k(s) = 1, x_k(t) =1, \exists r\in [s,t], x_k(r) = 0\big)\leqslant P(x_k(s)=1,y_k(t)=1).$$

Nous considérons la première instant de saut de $x_k$, soit 
$$T = \inf\{ r> s, x_k(s) =0 \}
$$
et nous conditionnons la première probabilité par $\{T=r\}$, et nous utilisons la propriété de Markov forte qui donne:
\begin{align*}&P\big(x_k(s) = 1, x_k(t) =1, \exists r\in [s,t], x_k(r) = 0\big) \\
&= E\left[\mathbf{1}_{x_k(s) = 1}\mathbf{1}_{x_k(t) = 1}\mathbf{1}_{T<t}\right] \\
&= E\left[E[\mathbf{1}_{x_k(s) = 1}\mathbf{1}_{x_k(t) = 1}\mathbf{1}_{T<t}| T=r]\right] \\
&= E\left[\mathbf{1}_{x_k(s) = 1}\mathbf{1}_{T = r}E[\mathbf{1}_{x_k(t) = 1}| T=r]\right] \\
& = E\left[\mathbf{1}_{x_k(s) = 1}\mathbf{1}_{T <t}P_0(x'_k(t-r) =1)\right] \\
& \leqslant E\left[\mathbf{1}_{x_k(s) = 1}\mathbf{1}_{T <t}\right]P(y_k(t-r) =1)\\
& = P(x_k(s) = 1, T< t) P(y_k(t)=1) \\
& \leqslant P(x_k(s)=1, y_k(t) = 1).
\end{align*}
Avec $x'_k$ le processus démarré à $T$. Nous avons $P_0(x'_k(t-r) = 1) \leqslant P(y_k(t-r) = 1)$ car $\{x_k(t-r) = 1\}$ est un événement croissant. Nous avons $P(y_k(t-r)= 1) = P(y_k(t)+1)$ car $y_k$ est stationnaire à l'équilibre.

Il reste le cas $e_k$ pivot pour $B$ mais pas pour $A$. Nous avons maintenant $x_k(t) = 1$, nous posons 
$$ x'=(x_1,\dots,x_{k-1},f_k,x_{k+1},\dots,x_{|\Lambda(l,h)|})
$$
$$y' =(y_1,\dots,y_{k-1},x'_k,y_{k+1},\dots,y_{|\Lambda(l,h)|})
$$
où $f_k:[s,t]\rightarrow\{0,1\}$ une fonction càdlàg et $x'_k$ une copie indépendante de $x_k$. Nous avons $x'\times y' \in A'\circ B'_k$, et $P_{12}(x\times y \in A'\circ B'_k; e_k \text{ pivot }B) = \frac{1}{p}P_{12}(x\times y \in A'\circ B'_{k-1}; e_k \text{ pivot }B)$.

Nous montrons maintenant $P(e_k \text{ pivot } A,B) \leqslant P(e_k \text{ pivot } B)$ quand $p$ proche de 1. En effet, 
\begin{multline*}
P(e_k \text{ pivot } A,B)\leqslant P(x_k(s) = 1, x_k(t) =1, \exists r\in [s,t], x_k(r) = 0)\\
 \leqslant 1- \exp(-(1-p)(t-s))
\end{multline*}
$$P(e_k \text{ pivot }B) \geqslant P(e_k \text{ ne change pas d'état}) \geqslant \exp(-(t-s))
$$
Pour $p$ assez proche de 1, nous avons l'inégalité demandée. Nous avons donc
\begin{align*}
P_{12}(A'\circ B'_{k-1}) =& P_{12}(A'\circ B'_{k-1};e_k \text{ pivot } A,B) + P_{12}(A'\circ B'_{k-1};e_k \text{ pivot } B) \\
  &+ P_{12}(A'\circ B'_{k-1};e_k \text{ non pivot })\\
\leqslant &  p P_{12}(A'\circ B'_{k-1};e_k \text{ pivot } A,B) + \frac{1}{p}P_{12}(A'\circ B'_{k-1};e_k \text{ pivot } B) \\
  &+ P_{12}(A'\circ B'_{k-1};e_k \text{ non pivot })\\
\leqslant & P_{12}(A'\circ B'_{k};e_k \text{ pivot } A,B) + P_{12}(A'\circ B'_{k};e_k \text{ pivot } B) \\
  &+ P_{12}(A'\circ B'_{k};e_k \text{ non pivot })\\
  \leqslant & P_{12}(A'\circ B'_{k})
\end{align*}
Avec une récurrence sur $k$, nous avons l'inégalité demandée dans la proposition.
\end{proof}

Nous avons déjà qu'un chemin fermé admet la propriété de décroissance exponentielle en fonction de son cardinal dans la phase surcritique, nous montrons maintenant une propriété similaire pour la percolation dynamique sur son comportement temporel: 

\begin{decexp}
Soit $p>\frac{1}{2}$, $c$ un chemin fermé à l'instant $s$ de cardinal $m$, soit $P_{c,c'}$ la probabilité qu'il ne soit pas disjoint d'un chemin $c'$ à l'instant $t>s$, nous avons $P_{c,c'}\leqslant me^{-\gamma(t-s)}$ avec $\gamma$ une constante indépendante de $c$ et de $c'$.
\end{decexp}
\begin{proof}
nous considérons une arête $(x,y)$ fermé et une modification locale $M$ pour l'ouvrir suivante: nous fermons deux arêtes perpendiculaires de même côté qui contiennent respectivement un sommet $x,y$ et l'arête qui relie ces deux arêtes; ensuite nous ouvrons l'arête $(x,y)$. Chaque étape de la modification est déterminée par une horloge exponentielle et elle respecte la condition de $T\nleftrightarrow B$. Entre $[0,1]$, cette modification a une probabilité positive $r$ pour se réaliser. Nous en déduisons qu'il existe une constante $\gamma$ pour qu'entre $[s,t]$, la probabilité que cette modification ne se réalise pas est inférieur à $e^{-\gamma(t-s)}$. 

Nous considérons maintenant deux arêtes $e,e'$ voisines, nous pouvons choisir les arêtes que nous modifions pour que $e,e'$ soient modifiées indépendamment. En effet, si les $e,e'$ sont colinéaires alors nous effectuons les modifications à différents côtés; si $e,e'$ sont perpendiculaires, alors nous effectuons la modification à l'extérieur de l'angle formé par $e,e'$, voir figure \ref{fig:mod}.

\begin{figure}[h]

\begin{minipage}{0.45\linewidth}
\center
\begin{tikzpicture}
\draw[very thin] (0,0) grid (4,4);
\node[below] at (1.5,2) {$e$};
\node[above] at (2.5,2) {$e'$};
\draw[red,very thick] (1,2) -- (1,3) -- (2,3) -- (2,1) -- (3,1) -- (3,2);
\end{tikzpicture}
\end{minipage}
\hfill
\begin{minipage}{0.45\linewidth}
\center
\begin{tikzpicture}
\draw[very thin] (0,0) grid (4,4);
\node[right] at (2,2.5) {$e$};
\node[above] at (2.5,2) {$e'$};
\draw[red,very thick] (2,3) -- (1,3) -- (1,2) -- (2,2) -- (2,1) -- (3,1) -- (3,2);
\end{tikzpicture}
\end{minipage}
\caption{deux arêtes voisines et les arêtes à modifier en rouge}
\label{fig:mod}

\end{figure}

Nous avons donc 
\begin{align*}
 P(c \coprod c') &\geqslant P(\forall e \in c, M \text{ se réalise sur } e) \\
 & \geqslant (1-e^{-\gamma (t-s)})^m \geqslant 1-me^{-\gamma (t-s)}
\end{align*} 
\end{proof}

Nous introduisons la notion de space-time chemin, en prolongeant la connexion dans le temps, c'est-à-dire si une arête $e=(x,y)$ reste ouverte (resp. fermée) entre $s$ et $t$ alors $(x,s)\longleftrightarrow (y,t)$ par un space-time chemin ouvert (resp.fermé). Nous montrons aussi une décroissance exponentielle avec les space-time chemin.

\begin{stc} \label{stc} Soit $n\in \mathbb{N}$, $t>0$, $p> \frac{1}{2}$, soit $ A(n,t)$ l'événement $(O,0) \longleftrightarrow (n\mathbf{e}_1,t)$ par un space-time chemin fermé ou $\mathbf{e}_1 = (1,0)$, alors $\exists \gamma(p,t),$ une constante qui dépend de $p,t$ tel que $P(A(n,t)) \sim e^{-\gamma(p,t)n}$.
\end{stc}

\begin{proof}
Nous montrons cette équivalence par le lemme sous-additif. En fait, 
\begin{align*}
P((O,0)\longleftrightarrow ((n+m)\mathbf{e}_1,s+t)) &\geqslant P((O,0)\longleftrightarrow (n\mathbf{e}_1,s))P((n\mathbf{e}_1,s)\longleftrightarrow ((m+n)\mathbf{e}_1,s+t))\\
& \leqslant P((O,0)\longleftrightarrow (n\mathbf{e}_1,s))P((O,0)\longleftrightarrow (m\mathbf{e}_1,t))
\end{align*}
Car nous avons l'invariance par translation. Nous concluons avec le lemme sous additif.
\end{proof}

\begin{comment}
Nous étudions la constante $\gamma(p,t)$ et nous avons la proposition suivante:
\begin{tension}$\frac{\gamma(p,t)}{\gamma(p,0)} \rightarrow 1$ quand $p\rightarrow 1$.
\end{tension}
\begin{proof}
Nous utilisons le fait que $\lim_n \frac{1}{n}\ln P(O\longleftrightarrow n\mathbf{e}_1) = \lim_n \frac{1}{n}\ln P(O\longleftrightarrow \partial \Lambda_n)$ et nous considérons une boite de taille $n$. Nous remarquons le nombre d'arête qui se ferme entre $[0,t]$ est borné par une loi de Poisson de paramètre $n^2(1-p)t$. Nous notons $N$ le nombre de bout d'un space-time chemin, c'est-à-dire le nombre de chemins fermés tel que $$ (O,0) = (x_1,t_1) \longleftrightarrow (x_1,t_2) \longleftrightarrow (x_2,t_2)\dots \longleftrightarrow (x_N,t_N) \longleftrightarrow (\partial\Lambda_n,t).
$$
Nous avons donc $N$ est borné par une variable aléatoire de loi de Poisson de paramètre $n^2(1-p)t$. Quand $p\rightarrow 1$, nous pouvons rendre $N$ d'ordre de 1 avec une probabilité proche de 1 et nous avons le résultat.
\end{proof}
\end{comment}
\section{La probabilité d'une influence du bord}

Nous montrons maintenant que la probabilité d'avoir une influence du bord sur une arête à l'intérieur de la boîte. 
\begin{cvg}
Soit $p\geqslant \frac{1}{2}$, $\bar{e}$ l'arête au centre de la boîte $\Lambda(l,h), \Lambda(l,h)'$ décrit dans la proposition \ref{chaine}, il existe une constante $\lambda> 0$ tel que $P(Y(\bar{e})\neq Y'(\bar{e})) \leqslant e^{-\lambda l} P(T\nleftrightarrow B \text{ entre }[0,t])$.
\end{cvg}

\begin{proof}

Nous utilisons les notations de la proposition \ref{chaine} et son corollaire.
D'après la proposition \ref{chaine}, nous savons qu'il existe une suite d'arêtes $\varepsilon_1,\varepsilon_n$ qui sont reliées l'une après l'autre à différentes instants. Nous notons $x_i,y_i$ les extrémités de $c_i^*$ chemin fermé dans le graphe dual et de $C_i$ le cut, nous notons aussi $k_i$ le cardinal de $p_i^1\cup p_i^2$. Nous séparons la suite en différentes sous suites selon l'indépendance, plus précisément, si $c^*_j$ et $c^*_{j+1}$  est de l'occurrence disjointe, alors nous coupons la suite à l'indice $j$. Ainsi, nous obtenons les indices $j_1,\dots,j_r$ telles que $\forall 1\leqslant u\leqslant r, c^*_{j_u}$ et $c^*_{j_{u+1}}$ sont disjoints, $\forall j_k \leqslant v \leqslant j_{k+1}$, $c_k^*$ et $c_{k+1}^*$ ne sont pas disjoints.

\begin{align*}
P(Y(\bar{e})\neq Y'(\bar{e})) =& P(\exists \varepsilon_1,\dots,\varepsilon_n, \forall i, \exists p_i^1,p_i^2, \varepsilon_{i-1}\overset{p_i^1,p_i^2}{\longleftrightarrow}\varepsilon_i) \\
 \leqslant &P(\exists x_1,\dots,x_n,y_1,\dots,y_n c^*_1,\dots,c^*_n,C_1,\dots,C_n) \\\
 \leqslant &\sum_{j_1,\dots,j_r}\prod_{1\leqslant k \leqslant r} P\left(\begin{array}{c}
 \exists x_{j_{k-1}+1},\dots,x_{j_k},\\
 y_{j_{k-1}+1},\dots,y_{j_k},\\
 c_{j_{k-1}+1},\dots,c^*_{j_k},\\
 C_{j_{k-1}+1},\dots,C_{j_k} \text{ cut},\\
 \forall j_{k-1}+1 \leqslant m \leqslant j_k, x_m\overset{c^*_m}{\longleftrightarrow} y_m, 
 c^*_m\circ C_m
 \end{array}
 \right)
\end{align*}
Nous utilisons la proposition \ref{stc} pour majorer chaque terme du produit. Or les $\forall j_{k-1}+1 \leqslant c^*_m \leqslant j_k$, les $c^*_m$ ne sont pas d'occurrence disjointe, nous avons un space-time chemin $\sigma_k$ qui relie $x_{j_{k-1}+1}$ et $y_{j_k}$. Donc elle est bornée par $$ \displaystyle |\sigma_k|^4 e^{-\gamma(p,t_{j_k}-t_{j_{k-1}+1})|\sigma_k|} P(\exists C_{j_{k-1}+1},\dots,C_{j_k} \text{ cut})$$ car $x_{j_{k-1}+1}$ et $y_{j_k}$ sont dans un carré de taille inférieure à $ |\sigma_k|$. Or $\exists \delta>0$ tel que $\forall x>1, x^4e^{-x} \leqslant e^{-\delta x}$, nous avons 
$$|\sigma_k|^4 e^{-\gamma(p,t_{j_k}-t_{j_{k-1}+1})|\sigma_k|} \leqslant |\sigma_k|^4 e^{-\gamma(p,t)|\sigma_k|} \leqslant e^{-\delta\gamma(p,t)|\sigma_k|}$$

Nous partons de $\bar{e}$, $x_{j_{n-1}+1}$ est de distance inférieure à $2|\sigma_r|$ de $\bar{e}$, nous avons donc 
\begin{align*}\sum_{1=j_1<\dots < j_r = n} &\prod_{1\leqslant k \leqslant r} e^{-\delta \gamma(p,t)|\sigma_k|} P(\exists C_{j_{k-1}+1},\dots,C_{j_k} \text{ cut}) \\
\leqslant & P(\exists C_1,\dots,C_n \text{ cut})\sum_{1=j_1<\dots < j_{r-1}} 4|\sigma_r|^2e^{-\delta \gamma(p,t)|\sigma_r|}\prod_{1\leqslant k \leqslant r-1}e^{-\delta \gamma(p,t)|\sigma_k|} \\
\leqslant & P(\exists C_1,\dots,C_n \text{ cut})\sum_{1=j_1<\dots < j_{r-1}} 4e^{-\delta^2 \gamma(p,t)|\sigma_r|}\prod_{1\leqslant k \leqslant r-1}e^{-\delta \gamma(p,t)|\sigma_k|} \\
\leqslant & 4^r e^{-\delta^2 \gamma(p,t)\sum_1^r|\sigma_k|}P(\exists C_1,\dots,C_n \text{ cut})
\end{align*}
Enfin, $r$ est borné par une variable aléatoire de loi de Poisson de paramètre $l^2 (1-p) t$, nous avons le résultat.
\end{proof}

\end{document}